\documentclass{article}
\usepackage{graphicx}
\usepackage{listings}
\usepackage{xcolor}

% Định nghĩa ngôn ngữ Protobuf cho lstlisting
\lstdefinelanguage{protobuf}{
  keywords={syntax, service, rpc, message, returns},
  keywordstyle=\color{blue}\bfseries,
  morestring=[b]",
  stringstyle=\color{red},
  comment=[l]{//},
  morecomment=[s]{/*}{*/},
  commentstyle=\color{green!70!black}\itshape,
  numbers=left,
  numberstyle=\tiny\color{gray},
  breaklines=true, % Tự động ngắt dòng dài
  postbreak=\mbox{\textcolor{red}{$\hookrightarrow$}\space}, % Ký hiệu ngắt dòng
  frame=single, % Thêm khung
  showspaces=false,
  showstringspaces=false,
  tabsize=2
}

% Cài đặt lstlisting chung
\lstset{
  basicstyle=\ttfamily\footnotesize,
  captionpos=b
}

\title{02.RPC.File.Transfer}
\author{
    Le Quoc Trung \and 
    Nguyen Kien Trung \and 
    Nguyen Dinh Tung \and 
    Nguyen Huy Tung \and 
    Nguyen Duc Bao Minh
}
\date{\today}

\begin{document}

\maketitle

\section*{Introduction}
This report documents the design and implementation of an RPC-based file transfer system. The system is implemented using gRPC and consists of server and client components.

\section*{RPC Service Design}
The RPC service is defined in a Protocol Buffers file (`file_transfer.proto`). It contains two main services:
\begin{itemize}
    \item \textbf{UploadFile}: Streams file chunks from the client to the server.
    \item \textbf{DownloadFile}: Streams file chunks from the server to the client.
\end{itemize}

The service definition in Protocol Buffers:
\begin{lstlisting}[language=protobuf, caption={Protocol Buffers Definition}]
syntax = "proto3";

service FileTransferService {
  rpc UploadFile (stream FileChunk) returns (UploadStatus);
  rpc DownloadFile (FileRequest) returns (stream FileChunk);
}

message FileChunk {
  bytes content = 1;
  string filename = 2;
}

message FileRequest {
  string filename = 1;
}

message UploadStatus {
  string message = 1;
  bool success = 2;
}
\end{lstlisting}

\section*{System Architecture}
The system consists of two main components:
\begin{itemize}
    \item \textbf{Server}: Handles file upload and download requests.
    \item \textbf{Client}: Sends file chunks to the server and receives file chunks during downloads.
\end{itemize}

\begin{figure}[h!]
\centering
\includegraphics[width=0.8\textwidth]{archit.png}  
\caption{System Architecture}
\end{figure}

\section*{Implementation}
The implementation uses gRPC for communication. Below are code snippets demonstrating the core functionality.

\subsection*{Server Implementation}
The server processes file upload and download requests:
\begin{lstlisting}[language=python, caption={Server Implementation}, frame=single]
import grpc
from concurrent import futures
import file_transfer_pb2_grpc
import file_transfer_pb2

class FileTransferService(file_transfer_pb2_grpc.FileTransferServiceServicer):
    def UploadFile(self, request_iterator, context):
        for chunk in request_iterator:
            with open(chunk.filename, "ab") as f:
                f.write(chunk.content)
        return file_transfer_pb2.UploadStatus(
            message="File uploaded successfully", success=True)

    def DownloadFile(self, request, context):
        try:
            with open(request.filename, "rb") as f:
                while True:
                    chunk = f.read(1024)
                    if not chunk:
                        break
                    yield file_transfer_pb2.FileChunk(
                        content=chunk, filename=request.filename)
        except FileNotFoundError:
            context.set_code(grpc.StatusCode.NOT_FOUND)
            context.set_details(f"File '{request.filename}' not found.")

server = grpc.server(futures.ThreadPoolExecutor(max_workers=10))
file_transfer_pb2_grpc.add_FileTransferServiceServicer_to_server(
    FileTransferService(), server)
server.add_insecure_port("[::]:50051")
server.start()
server.wait_for_termination()
\end{lstlisting}

\subsection*{Client Implementation}
The client uploads and downloads files using the defined gRPC services:
\begin{lstlisting}[language=python, caption={Client Implementation}, frame=single]
import grpc
import file_transfer_pb2_grpc
import file_transfer_pb2

def upload_file(stub, filename):
    def generate_chunks():
        with open(filename, "rb") as f:
            while True:
                chunk = f.read(1024)
                if not chunk:
                    break
                yield file_transfer_pb2.FileChunk(
                    content=chunk, filename=filename)
    response = stub.UploadFile(generate_chunks())
    print(response.message)

def download_file(stub, filename):
    response = stub.DownloadFile(file_transfer_pb2.FileRequest(
        filename=filename))
    with open(f"downloaded_{filename}", "wb") as f:
        for chunk in response:
            f.write(chunk.content)

if __name__ == "__main__":
    channel = grpc.insecure_channel("localhost:50051")
    stub = file_transfer_pb2_grpc.FileTransferServiceStub(channel)
    upload_file(stub, "example.txt")
    download_file(stub, "example.txt")
\end{lstlisting}

\section*{Conclusion}
The gRPC-based file transfer system is robust and efficient, leveraging Protocol Buffers for communication. This modular design can be extended for additional functionalities like file metadata or authentication.

\end{document}
